\begin{problem}{Трудная задача}{стандартный ввод}{стандартный вывод}{1 секунда}{512 мегабайт}

Дан неориентированный граф, состоящий из $3n$ вершин и \textbf{ровно} $3n$ рёбер. Независимым множеством назовём множество вершин такое, что между любыми двумя вершинами в нём нет ребра.

Вам интересно, можно ли найти в данном графе независимое множество, состоящее из $n$ вершин.


\InputFile
В первой строке задано целое число $n$ ($1 \leq n \leq 100\,000$).

В следующих $3n$ строках задано описание рёбер графа. Каждое ребро задано парой целых чисел $u_i, v_i$ ($1 \leq u_i, v_i \leq 3n$)~--- номерами вершин, которые оно соединяет.

Гарантируется, что в графе нет петель и кратных рёбер.


\OutputFile
В случае, если в графе нет независимого множества размера $n$, выведите <<\texttt{No}>>. В противном случае в первой строке выведите <<\texttt{Yes}>>, а во второй~--- $n$ различных целых чисел~--- номера вершин в независимом множестве.

Если существует несколько решений, выведите любое из них.


\Examples

\begin{example}
\exmpfile{example.01}{example.01.a}%
\exmpfile{example.02}{example.02.a}%
\end{example}

\Note
В первом тесте из условия граф выглядит так:


\begin{center}
\includegraphics[height=2cm]{graph1.eps}
\end{center}


Во втором тесте из условия граф выглядит так:

\begin{center}
\includegraphics[height=3cm]{graph2.eps}
\end{center}


\end{problem}

