\textit{Автор и разработчик задачи: Степан Стёпкин}

Воспользуемся следующим алгоритмом. Изначально добавим все вершины в независимое множество. На первом шаге будем удалять вершины из независимого множества пока они соединены хотя бы с двумя вершинами множества. На втором шаге будем удалять все вершины, которые соединены хотя бы с одной вершиной независимого множества.

Очевидно, что множество, полученное в конце работы алгоритма, действительно является независимым. Покажем, что оно имеет размер хотя бы $n$.

Предположим, что после первого шага работы алгоритма в множестве осталось $F$ вершин, соединённых $M$ рёбрами. Во-первых, $2(3n - F) \leq 3n - M$, поскольку при удалении каждой вершины было удалено хотя бы два ребра. Во-вторых, $2M \leq F$, поскольку каждая вершина из множества соединена не более, чем с одной вершиной из множества.

Сложим эти неравенства; после упрощения получим $F - M \geq n$. Заметим, что $F - M$~--- это в точности размер независимого множества в конце, поскольку при удалении каждой вершины на втором шаге удаляется ровно одно ребро.

Данный алгоритм нетрудно реализовать так, чтобы он имел сложность $O(n)$ или $O(n \log n)$.