\begin{problem}{Кандибобер}{стандартный ввод}{стандартный вывод}{1 секунда}{256 мегабайт}

Шахматная фигура \textbf{Кандибобер}, ходит как конь, если она стоит на белой клетке, и как слон, если на черной.

У вас есть бесконечное шахматное поле. Вам известно что Кандибобер стоит на поле с координатами $(X_1, Y_1)$. Вам нужно за минимальное количество ходов можно попасть в клетку $(X_2, Y_2)$.

Клетка с координатами $(0, 0)$ - белая.


\InputFile
В первой строке даны четыре числа - $X_1, Y_1, X_2, Y_2$($-10^9 \leq X_1, Y_1, X_2, Y_2 \leq 10^9$).

\OutputFile
В первой строке выведите единственное число M - минимальное число ходов.
Далее в M строках выведите координаты клеток, на которые должен совершиться ход. Все координаты должны быть не более чем $2 \cdot 10^9$ по модулю. Гарантируется что оптимальное решение с такими координатами существует. Последней клеткой должна быть $(X_2, Y_2)$.
Если попасть в данную клетку невозможно, в единственной выведите \textbf{-1}.

\Examples

\begin{example}
\exmpfile{example.01}{example.01.a}%
\exmpfile{example.02}{example.02.a}%
\exmpfile{example.03}{example.03.a}%
\end{example}

\Note
Ваше решение наберет не менее 12 баллов, если вы верно определите все случаи, когда попасть в клетку невозможно.

Ваше решение наберет не менее 34 баллов, если вы верно решите задачу, когда гарантируется, что ответ достижим за не более чем 1 передвижение фигурой.

Ваше решение наберет не менее 38 баллов, если вы верно решите задачу для $-100 \leq X_1, Y_1, X_2, Y_2 \leq 100$.

\end{problem}

