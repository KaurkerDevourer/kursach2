Недавно Ваня в школе проходил геометрические прогрессии. Им задали посчитать сумму геометрической прогрессии последовательности $\frac{a^i}{a + 1}$. Чтобы не считать ручками, он написал себе программу. Но программа умеет работать только с целыми числами. Таким образом, для данных $a$ и $n$ программа считала сумму $[\frac{1}{a+1}] + [\frac{a}{a+1}] + [\frac{a^2}{a+1}] + ... + [\frac{a^n}{a+1}]$. Здесь [$a$]~--- целая часть числа $a$, то есть максимальное целое число $b$ такое, что $b \leq a$. Смоделируйте работу программы для данных $a$ и $n$.