\begin{problem}{Округлить}{стандартный ввод}{стандартный вывод}{1 секунда}{256 мегабайт}

Недавно Ваня в школе проходил геометрические прогрессии. Им задали посчитать сумму геометрической прогрессии последовательности $\frac{a^i}{a + 1}$. Чтобы не считать ручками, он написал себе программу. Но программа умеет работать только с целыми числами. Таким образом, для данных $a$ и $n$ программа считала сумму $[\frac{1}{a+1}] + [\frac{a}{a+1}] + [\frac{a^2}{a+1}] + ... + [\frac{a^n}{a+1}]$. Здесь [$a$]~--- целая часть числа $a$, то есть максимальное целое число $b$ такое, что $b \leq a$. Смоделируйте работу программы для данных $a$ и $n$.

\InputFile
В единственной строке вам даны два целых числа~--- $a$, $n$ ($1 \leq a, n \leq 10^9$)

\OutputFile
Выведите единственное число~--- ответ на задачу по модулю $10^9 + 7$

\Examples

\begin{example}
\exmpfile{example.01}{example.01.a}%
\exmpfile{example.02}{example.02.a}%
\exmpfile{example.03}{example.03.a}%
\end{example}

\end{problem}

